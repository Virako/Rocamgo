%
% API Documentation for API Documentation
% Module src.kifu
%
% Generated by epydoc 3.0.1
% [Sun Sep  9 21:09:35 2012]
%

%%%%%%%%%%%%%%%%%%%%%%%%%%%%%%%%%%%%%%%%%%%%%%%%%%%%%%%%%%%%%%%%%%%%%%%%%%%
%%                          Module Description                           %%
%%%%%%%%%%%%%%%%%%%%%%%%%%%%%%%%%%%%%%%%%%%%%%%%%%%%%%%%%%%%%%%%%%%%%%%%%%%

    \index{src \textit{(package)}!src.kifu \textit{(module)}|(}
\section{Module src.kifu}

    \label{src:kifu}

%%%%%%%%%%%%%%%%%%%%%%%%%%%%%%%%%%%%%%%%%%%%%%%%%%%%%%%%%%%%%%%%%%%%%%%%%%%
%%                               Variables                               %%
%%%%%%%%%%%%%%%%%%%%%%%%%%%%%%%%%%%%%%%%%%%%%%%%%%%%%%%%%%%%%%%%%%%%%%%%%%%

  \subsection{Variables}

    \vspace{-1cm}
\hspace{\varindent}\begin{longtable}{|p{\varnamewidth}|p{\vardescrwidth}|l}
\cline{1-2}
\cline{1-2} \centering \textbf{Name} & \centering \textbf{Description}& \\
\cline{1-2}
\endhead\cline{1-2}\multicolumn{3}{r}{\small\textit{continued on next page}}\\\endfoot\cline{1-2}
\endlastfoot\raggedright \_\-\_\-p\-a\-c\-k\-a\-g\-e\-\_\-\_\- & \raggedright \textbf{Value:} 
{\tt \texttt{'}\texttt{src}\texttt{'}}&\\
\cline{1-2}
\end{longtable}


%%%%%%%%%%%%%%%%%%%%%%%%%%%%%%%%%%%%%%%%%%%%%%%%%%%%%%%%%%%%%%%%%%%%%%%%%%%
%%                           Class Description                           %%
%%%%%%%%%%%%%%%%%%%%%%%%%%%%%%%%%%%%%%%%%%%%%%%%%%%%%%%%%%%%%%%%%%%%%%%%%%%

    \index{src \textit{(package)}!src.kifu \textit{(module)}!src.kifu.Kifu \textit{(class)}|(}
\subsection{Class Kifu}

    \label{src:kifu:Kifu}
Clase para crear un fichero .sgf y guardar la partida.


%%%%%%%%%%%%%%%%%%%%%%%%%%%%%%%%%%%%%%%%%%%%%%%%%%%%%%%%%%%%%%%%%%%%%%%%%%%
%%                                Methods                                %%
%%%%%%%%%%%%%%%%%%%%%%%%%%%%%%%%%%%%%%%%%%%%%%%%%%%%%%%%%%%%%%%%%%%%%%%%%%%

  \subsubsection{Methods}

    \label{src:kifu:Kifu:__init__}
    \index{src \textit{(package)}!src.kifu \textit{(module)}!src.kifu.Kifu \textit{(class)}!src.kifu.Kifu.\_\_init\_\_ \textit{(method)}}

    \vspace{0.5ex}

\hspace{.8\funcindent}\begin{boxedminipage}{\funcwidth}

    \raggedright \textbf{\_\_init\_\_}(\textit{self}, \textit{player1}={\tt \texttt{'}\texttt{j1}\texttt{'}}, \textit{player2}={\tt \texttt{'}\texttt{j2}\texttt{'}}, \textit{handicap}={\tt 0}, \textit{path}={\tt \texttt{'}\texttt{sgf}\texttt{'}}, \textit{rank\_player1}={\tt \texttt{'}\texttt{20k}\texttt{'}}, \textit{rank\_player2}={\tt \texttt{'}\texttt{20k}\texttt{'}})

    \vspace{-1.5ex}

    \rule{\textwidth}{0.5\fboxrule}
\setlength{\parskip}{2ex}
    Inicializamos configuración del archivo sgf. :param  player1: nombre 
    del jugador 1 :type  player1: str :keyword  player1: j1 por defecto 
    :param  player2: nombre del jugador 2 :type  player2: str :keyword  
    player2: j2 por defecto :param  handicap: handicap dado en la partida 
    :type  handicap: int :keyword  handicap: ninguno por defecto (0) :param
    path: ruta relativa donde guardamos el fichero :type  path: str 
    :keyword  path: carpeta sgf por defecto :param  rank\_player1: rango 
    del jugador 1 :type  rank\_player1: str :keyword  rank\_player1: 20k 
    por defecto, nivel de inicio en el go :param  rank\_player2: rango del 
    jugador 2 :type  rank\_player2: str :keyword  rank\_player2: 20k por 
    defecto, nivel de inicio en el go

\setlength{\parskip}{1ex}
    \end{boxedminipage}

    \label{src:kifu:Kifu:add_stone}
    \index{src \textit{(package)}!src.kifu \textit{(module)}!src.kifu.Kifu \textit{(class)}!src.kifu.Kifu.add\_stone \textit{(method)}}

    \vspace{0.5ex}

\hspace{.8\funcindent}\begin{boxedminipage}{\funcwidth}

    \raggedright \textbf{add\_stone}(\textit{self}, \textit{pos}, \textit{color})

    \vspace{-1.5ex}

    \rule{\textwidth}{0.5\fboxrule}
\setlength{\parskip}{2ex}
    Añadir piedra al sgf. :param pos: posición de la piedra :type pos: 
    tuple :param color: color de la piedra :type color: int

\setlength{\parskip}{1ex}
    \end{boxedminipage}

    \label{src:kifu:Kifu:end_file}
    \index{src \textit{(package)}!src.kifu \textit{(module)}!src.kifu.Kifu \textit{(class)}!src.kifu.Kifu.end\_file \textit{(method)}}

    \vspace{0.5ex}

\hspace{.8\funcindent}\begin{boxedminipage}{\funcwidth}

    \raggedright \textbf{end\_file}(\textit{self})

    \vspace{-1.5ex}

    \rule{\textwidth}{0.5\fboxrule}
\setlength{\parskip}{2ex}
    Cerrar el fichero y dejarlo listo para poder abrirlo.

\setlength{\parskip}{1ex}
    \end{boxedminipage}

    \index{src \textit{(package)}!src.kifu \textit{(module)}!src.kifu.Kifu \textit{(class)}|)}
    \index{src \textit{(package)}!src.kifu \textit{(module)}|)}
