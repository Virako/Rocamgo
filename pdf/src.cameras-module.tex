%
% API Documentation for API Documentation
% Module src.cameras
%
% Generated by epydoc 3.0.1
% [Wed Sep 12 02:49:43 2012]
%

%%%%%%%%%%%%%%%%%%%%%%%%%%%%%%%%%%%%%%%%%%%%%%%%%%%%%%%%%%%%%%%%%%%%%%%%%%%
%%                          Module Description                           %%
%%%%%%%%%%%%%%%%%%%%%%%%%%%%%%%%%%%%%%%%%%%%%%%%%%%%%%%%%%%%%%%%%%%%%%%%%%%

    \index{src \textit{(package)}!src.cameras \textit{(module)}|(}
\section{Module src.cameras}

    \label{src:cameras}

%%%%%%%%%%%%%%%%%%%%%%%%%%%%%%%%%%%%%%%%%%%%%%%%%%%%%%%%%%%%%%%%%%%%%%%%%%%
%%                               Variables                               %%
%%%%%%%%%%%%%%%%%%%%%%%%%%%%%%%%%%%%%%%%%%%%%%%%%%%%%%%%%%%%%%%%%%%%%%%%%%%

  \subsection{Variables}

    \vspace{-1cm}
\hspace{\varindent}\begin{longtable}{|p{\varnamewidth}|p{\vardescrwidth}|l}
\cline{1-2}
\cline{1-2} \centering \textbf{Name} & \centering \textbf{Description}& \\
\cline{1-2}
\endhead\cline{1-2}\multicolumn{3}{r}{\small\textit{continued on next page}}\\\endfoot\cline{1-2}
\endlastfoot\raggedright \_\-\_\-p\-a\-c\-k\-a\-g\-e\-\_\-\_\- & \raggedright \textbf{Value:} 
{\tt \texttt{'}\texttt{src}\texttt{'}}&\\
\cline{1-2}
\raggedright c\-a\-m\-e\-r\-a\- & \raggedright cámara seleccionada

            {\it (type=Capture)}&\\
\cline{1-2}
\raggedright c\-a\-m\-e\-r\-a\-s\- & \raggedright lista de cámaras

            {\it (type=list)}&\\
\cline{1-2}
\end{longtable}


%%%%%%%%%%%%%%%%%%%%%%%%%%%%%%%%%%%%%%%%%%%%%%%%%%%%%%%%%%%%%%%%%%%%%%%%%%%
%%                           Class Description                           %%
%%%%%%%%%%%%%%%%%%%%%%%%%%%%%%%%%%%%%%%%%%%%%%%%%%%%%%%%%%%%%%%%%%%%%%%%%%%

    \index{src \textit{(package)}!src.cameras \textit{(module)}!src.cameras.Cameras \textit{(class)}|(}
\subsection{Class Cameras}

    \label{src:cameras:Cameras}

Clase para abrir las cámaras disponibles en el ordenador.

%%%%%%%%%%%%%%%%%%%%%%%%%%%%%%%%%%%%%%%%%%%%%%%%%%%%%%%%%%%%%%%%%%%%%%%%%%%
%%                                Methods                                %%
%%%%%%%%%%%%%%%%%%%%%%%%%%%%%%%%%%%%%%%%%%%%%%%%%%%%%%%%%%%%%%%%%%%%%%%%%%%

  \subsubsection{Methods}

    \label{src:cameras:Cameras:__init__}
    \index{src \textit{(package)}!src.cameras \textit{(module)}!src.cameras.Cameras \textit{(class)}!src.cameras.Cameras.\_\_init\_\_ \textit{(method)}}

    \vspace{0.5ex}

\hspace{.8\funcindent}\begin{boxedminipage}{\funcwidth}

    \raggedright \textbf{\_\_init\_\_}(\textit{self})

\setlength{\parskip}{2ex}
\setlength{\parskip}{1ex}
    \end{boxedminipage}

    \label{src:cameras:Cameras:on_mouse}
    \index{src \textit{(package)}!src.cameras \textit{(module)}!src.cameras.Cameras \textit{(class)}!src.cameras.Cameras.on\_mouse \textit{(method)}}

    \vspace{0.5ex}

\hspace{.8\funcindent}\begin{boxedminipage}{\funcwidth}

    \raggedright \textbf{on\_mouse}(\textit{self}, \textit{event}, \textit{x}, \textit{y}, \textit{flags}, \textit{camera})

    \vspace{-1.5ex}

    \rule{\textwidth}{0.5\fboxrule}
\setlength{\parskip}{2ex}

Capturador de eventos de click de ratón.
:param event: Evento capturado.
:type event: int
:param x: posición x del ratón.
:type x: int
:param y: posición y del ratón.
:type y: int
:param camera: objeto Camera.
:type camera: Camera
\setlength{\parskip}{1ex}
    \end{boxedminipage}

    \label{src:cameras:Cameras:check_cameras}
    \index{src \textit{(package)}!src.cameras \textit{(module)}!src.cameras.Cameras \textit{(class)}!src.cameras.Cameras.check\_cameras \textit{(method)}}

    \vspace{0.5ex}

\hspace{.8\funcindent}\begin{boxedminipage}{\funcwidth}

    \raggedright \textbf{check\_cameras}(\textit{self}, \textit{num}={\tt 99})

    \vspace{-1.5ex}

    \rule{\textwidth}{0.5\fboxrule}
\setlength{\parskip}{2ex}

Comprueba las cámaras disponibles.
:param num: máximo número de cámaras a comprobar
:keyword num: el valor por defecto es 99, ya que en Linux es lo
permitido
:param num: int
:return: lista de cámaras disponibles
:rtype: list of Camera
\setlength{\parskip}{1ex}
    \end{boxedminipage}

    \label{src:cameras:Cameras:show_and_select_camera}
    \index{src \textit{(package)}!src.cameras \textit{(module)}!src.cameras.Cameras \textit{(class)}!src.cameras.Cameras.show\_and\_select\_camera \textit{(method)}}

    \vspace{0.5ex}

\hspace{.8\funcindent}\begin{boxedminipage}{\funcwidth}

    \raggedright \textbf{show\_and\_select\_camera}(\textit{self})

    \vspace{-1.5ex}

    \rule{\textwidth}{0.5\fboxrule}
\setlength{\parskip}{2ex}

Muestra las cámaras disponibles en ventanas y da la opción de
seleccionar una de ellas pulsando doble click.
:return: cámara seleccionada
:rtype: Camera
\setlength{\parskip}{1ex}
    \end{boxedminipage}

    \index{src \textit{(package)}!src.cameras \textit{(module)}!src.cameras.Cameras \textit{(class)}|)}
    \index{src \textit{(package)}!src.cameras \textit{(module)}|)}
