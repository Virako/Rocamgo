%
% API Documentation for API Documentation
% Module src.search_stones
%
% Generated by epydoc 3.0.1
% [Sun Sep  9 21:09:35 2012]
%

%%%%%%%%%%%%%%%%%%%%%%%%%%%%%%%%%%%%%%%%%%%%%%%%%%%%%%%%%%%%%%%%%%%%%%%%%%%
%%                          Module Description                           %%
%%%%%%%%%%%%%%%%%%%%%%%%%%%%%%%%%%%%%%%%%%%%%%%%%%%%%%%%%%%%%%%%%%%%%%%%%%%

    \index{src \textit{(package)}!src.search\_stones \textit{(module)}|(}
\section{Module src.search\_stones}

    \label{src:search_stones}

%%%%%%%%%%%%%%%%%%%%%%%%%%%%%%%%%%%%%%%%%%%%%%%%%%%%%%%%%%%%%%%%%%%%%%%%%%%
%%                               Functions                               %%
%%%%%%%%%%%%%%%%%%%%%%%%%%%%%%%%%%%%%%%%%%%%%%%%%%%%%%%%%%%%%%%%%%%%%%%%%%%

  \subsection{Functions}

    \label{src:search_stones:search_stones}
    \index{src \textit{(package)}!src.search\_stones \textit{(module)}!src.search\_stones.search\_stones \textit{(function)}}

    \vspace{0.5ex}

\hspace{.8\funcindent}\begin{boxedminipage}{\funcwidth}

    \raggedright \textbf{search\_stones}(\textit{img}, \textit{corners}, \textit{dp}={\tt 1.7})

    \vspace{-1.5ex}

    \rule{\textwidth}{0.5\fboxrule}
\setlength{\parskip}{2ex}
    Devuelve las circunferencias encontradas en una imagen. :param img: 
    imagen donde buscaremos las circunferencias :type img: IplImage :param 
    corners: lista de esquinas :type corners: list :param dp: profundidad 
    de búsqueda de círculos :type dp: int :keyword dp: 1.7 era el valor que
    mejor funcionaba. Prueba y error

\setlength{\parskip}{1ex}
    \end{boxedminipage}

    \label{src:search_stones:check_color_stone}
    \index{src \textit{(package)}!src.search\_stones \textit{(module)}!src.search\_stones.check\_color\_stone \textit{(function)}}

    \vspace{0.5ex}

\hspace{.8\funcindent}\begin{boxedminipage}{\funcwidth}

    \raggedright \textbf{check\_color\_stone}(\textit{pt}, \textit{radious}, \textit{img}, \textit{threshold}={\tt 190})

    \vspace{-1.5ex}

    \rule{\textwidth}{0.5\fboxrule}
\setlength{\parskip}{2ex}
    Devuelve el color de la piedra dado el centro y el radio de la piedra y
    una imagen. También desechamos las piedras que no sean negras o 
    blancas. :param pt: centro de la piedra :type pt: tuple :param radious:
    radio de la piedra :type radious: int :param img: imagen donde 
    comprobaremos el color de ciertos pixeles :type img: IplImage :param 
    threshold: umbral de blanco :type threshold: int :keyword threshold: 
    190 cuando hay buena luminosidad

\setlength{\parskip}{1ex}
    \end{boxedminipage}


%%%%%%%%%%%%%%%%%%%%%%%%%%%%%%%%%%%%%%%%%%%%%%%%%%%%%%%%%%%%%%%%%%%%%%%%%%%
%%                               Variables                               %%
%%%%%%%%%%%%%%%%%%%%%%%%%%%%%%%%%%%%%%%%%%%%%%%%%%%%%%%%%%%%%%%%%%%%%%%%%%%

  \subsection{Variables}

    \vspace{-1cm}
\hspace{\varindent}\begin{longtable}{|p{\varnamewidth}|p{\vardescrwidth}|l}
\cline{1-2}
\cline{1-2} \centering \textbf{Name} & \centering \textbf{Description}& \\
\cline{1-2}
\endhead\cline{1-2}\multicolumn{3}{r}{\small\textit{continued on next page}}\\\endfoot\cline{1-2}
\endlastfoot\raggedright \_\-\_\-p\-a\-c\-k\-a\-g\-e\-\_\-\_\- & \raggedright \textbf{Value:} 
{\tt \texttt{'}\texttt{src}\texttt{'}}&\\
\cline{1-2}
\end{longtable}

    \index{src \textit{(package)}!src.search\_stones \textit{(module)}|)}
