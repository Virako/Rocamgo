%
% API Documentation for API Documentation
% Module src.igs
%
% Generated by epydoc 3.0.1
% [Wed Sep 12 02:49:43 2012]
%

%%%%%%%%%%%%%%%%%%%%%%%%%%%%%%%%%%%%%%%%%%%%%%%%%%%%%%%%%%%%%%%%%%%%%%%%%%%
%%                          Module Description                           %%
%%%%%%%%%%%%%%%%%%%%%%%%%%%%%%%%%%%%%%%%%%%%%%%%%%%%%%%%%%%%%%%%%%%%%%%%%%%

    \index{src \textit{(package)}!src.igs \textit{(module)}|(}
\section{Module src.igs}

    \label{src:igs}

%%%%%%%%%%%%%%%%%%%%%%%%%%%%%%%%%%%%%%%%%%%%%%%%%%%%%%%%%%%%%%%%%%%%%%%%%%%
%%                               Variables                               %%
%%%%%%%%%%%%%%%%%%%%%%%%%%%%%%%%%%%%%%%%%%%%%%%%%%%%%%%%%%%%%%%%%%%%%%%%%%%

  \subsection{Variables}

    \vspace{-1cm}
\hspace{\varindent}\begin{longtable}{|p{\varnamewidth}|p{\vardescrwidth}|l}
\cline{1-2}
\cline{1-2} \centering \textbf{Name} & \centering \textbf{Description}& \\
\cline{1-2}
\endhead\cline{1-2}\multicolumn{3}{r}{\small\textit{continued on next page}}\\\endfoot\cline{1-2}
\endlastfoot\raggedright \_\-\_\-p\-a\-c\-k\-a\-g\-e\-\_\-\_\- & \raggedright \textbf{Value:} 
{\tt \texttt{'}\texttt{src}\texttt{'}}&\\
\cline{1-2}
\raggedright p\-w\-d\- & \raggedright password correspondiente al usuario de Igs

            {\it (type=str)}&\\
\cline{1-2}
\raggedright s\- & \raggedright socket para la conexión con el servidor

            {\it (type=socket)}&\\
\cline{1-2}
\raggedright u\-s\-e\-r\- & \raggedright usuario del servidor Igs

            {\it (type=str)}&\\
\cline{1-2}
\end{longtable}


%%%%%%%%%%%%%%%%%%%%%%%%%%%%%%%%%%%%%%%%%%%%%%%%%%%%%%%%%%%%%%%%%%%%%%%%%%%
%%                           Class Description                           %%
%%%%%%%%%%%%%%%%%%%%%%%%%%%%%%%%%%%%%%%%%%%%%%%%%%%%%%%%%%%%%%%%%%%%%%%%%%%

    \index{src \textit{(package)}!src.igs \textit{(module)}!src.igs.Igs \textit{(class)}|(}
\subsection{Class Igs}

    \label{src:igs:Igs}

Clase que se comunica con el servidor de IGS.

%%%%%%%%%%%%%%%%%%%%%%%%%%%%%%%%%%%%%%%%%%%%%%%%%%%%%%%%%%%%%%%%%%%%%%%%%%%
%%                                Methods                                %%
%%%%%%%%%%%%%%%%%%%%%%%%%%%%%%%%%%%%%%%%%%%%%%%%%%%%%%%%%%%%%%%%%%%%%%%%%%%

  \subsubsection{Methods}

    \label{src:igs:Igs:__init__}
    \index{src \textit{(package)}!src.igs \textit{(module)}!src.igs.Igs \textit{(class)}!src.igs.Igs.\_\_init\_\_ \textit{(method)}}

    \vspace{0.5ex}

\hspace{.8\funcindent}\begin{boxedminipage}{\funcwidth}

    \raggedright \textbf{\_\_init\_\_}(\textit{self}, \textit{user}={\tt \texttt{'}\texttt{rocamgo}\texttt{'}}, \textit{pwd}={\tt \texttt{'}\texttt{qwe}\texttt{'}})

    \vspace{-1.5ex}

    \rule{\textwidth}{0.5\fboxrule}
\setlength{\parskip}{2ex}

Inicializamos la conexión con el servidor y creamos un tablero de
aprendizaje dentro del servidor para comenzar a subir la partida.
:param user: usuario que se conectará al servidor
:type user: str
:param password: contraseña del usuario para conetarse al servidor
:type password: str
\setlength{\parskip}{1ex}
    \end{boxedminipage}

    \label{src:igs:Igs:add_stone}
    \index{src \textit{(package)}!src.igs \textit{(module)}!src.igs.Igs \textit{(class)}!src.igs.Igs.add\_stone \textit{(method)}}

    \vspace{0.5ex}

\hspace{.8\funcindent}\begin{boxedminipage}{\funcwidth}

    \raggedright \textbf{add\_stone}(\textit{self}, \textit{pos})

    \vspace{-1.5ex}

    \rule{\textwidth}{0.5\fboxrule}
\setlength{\parskip}{2ex}

Añadimos piedra al servidor.
:param pos: posición de la piedra a añadir
:type pos: tuple
\setlength{\parskip}{1ex}
    \end{boxedminipage}

    \label{src:igs:Igs:close}
    \index{src \textit{(package)}!src.igs \textit{(module)}!src.igs.Igs \textit{(class)}!src.igs.Igs.close \textit{(method)}}

    \vspace{0.5ex}

\hspace{.8\funcindent}\begin{boxedminipage}{\funcwidth}

    \raggedright \textbf{close}(\textit{self})

    \vspace{-1.5ex}

    \rule{\textwidth}{0.5\fboxrule}
\setlength{\parskip}{2ex}

Cerramos la conexión con el servidor.
\setlength{\parskip}{1ex}
    \end{boxedminipage}

    \index{src \textit{(package)}!src.igs \textit{(module)}!src.igs.Igs \textit{(class)}|)}
    \index{src \textit{(package)}!src.igs \textit{(module)}|)}
