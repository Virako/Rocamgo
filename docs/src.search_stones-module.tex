%
% API Documentation for API Documentation
% Módulo src.search_stones
%
% Generated by epydoc 3.0.1
% [Wed Sep 12 04:59:27 2012]
%

%%%%%%%%%%%%%%%%%%%%%%%%%%%%%%%%%%%%%%%%%%%%%%%%%%%%%%%%%%%%%%%%%%%%%%%%%%%
%%                          Módulo Description                           %%
%%%%%%%%%%%%%%%%%%%%%%%%%%%%%%%%%%%%%%%%%%%%%%%%%%%%%%%%%%%%%%%%%%%%%%%%%%%

    \index{src \textit{(package)}!src.search\_stones \textit{(module)}|(}
\section{Módulo src.search\_stones}

    \label{src:search_stones}

%%%%%%%%%%%%%%%%%%%%%%%%%%%%%%%%%%%%%%%%%%%%%%%%%%%%%%%%%%%%%%%%%%%%%%%%%%%
%%                               Funciones                               %%
%%%%%%%%%%%%%%%%%%%%%%%%%%%%%%%%%%%%%%%%%%%%%%%%%%%%%%%%%%%%%%%%%%%%%%%%%%%

  \subsection{Funciones}

    \label{src:search_stones:search_stones}
    \index{src \textit{(package)}!src.search\_stones \textit{(module)}!src.search\_stones.search\_stones \textit{(function)}}

    \vspace{0.5ex}

\hspace{.8\funcindent}\begin{boxedminipage}{\funcwidth}

    \raggedright \textbf{search\_stones}(\textit{img}, \textit{corners}, \textit{dp}={\tt 1.7})

    \vspace{-1.5ex}

    \rule{\textwidth}{0.5\fboxrule}
\setlength{\parskip}{2ex}
Devuelve las circunferencias encontradas en una imagen.

\setlength{\parskip}{1ex}
      \textbf{Parametros}
      \vspace{-1ex}

      \begin{quote}
        \begin{Ventry}{xxxxxxx}

          \item[img]


imagen donde buscaremos las circunferencias
            {\it (tipo=IplImage)}

          \item[corners]


lista de esquinas
            {\it (tipo=list)}

          \item[dp]


profundidad de búsqueda de círculos
            {\it (tipo=int)}

          \item[dp]


1.7 era el valor que mejor funcionaba. Prueba y error
            {\it (tipo=int)}

        \end{Ventry}

      \end{quote}

    \end{boxedminipage}

    \label{src:search_stones:check_color_stone}
    \index{src \textit{(package)}!src.search\_stones \textit{(module)}!src.search\_stones.check\_color\_stone \textit{(function)}}

    \vspace{0.5ex}

\hspace{.8\funcindent}\begin{boxedminipage}{\funcwidth}

    \raggedright \textbf{check\_color\_stone}(\textit{pt}, \textit{radious}, \textit{img}, \textit{threshold}={\tt 190})

    \vspace{-1.5ex}

    \rule{\textwidth}{0.5\fboxrule}
\setlength{\parskip}{2ex}
Devuelve el color de la piedra dado el centro y el radio de la piedra y una imagen. También desechamos las piedras que no sean negras o blancas.

\setlength{\parskip}{1ex}
      \textbf{Parametros}
      \vspace{-1ex}

      \begin{quote}
        \begin{Ventry}{xxxxxxxxx}

          \item[pt]


centro de la piedra
            {\it (tipo=tuple)}

          \item[radious]


radio de la piedra
            {\it (tipo=int)}

          \item[img]


imagen donde comprobaremos el color de ciertos pixeles
            {\it (tipo=IplImage)}

          \item[threshold]


umbral de blanco
            {\it (tipo=int)}

          \item[threshold]


190 cuando hay buena luminosidad
            {\it (tipo=int)}

        \end{Ventry}

      \end{quote}

    \end{boxedminipage}


%%%%%%%%%%%%%%%%%%%%%%%%%%%%%%%%%%%%%%%%%%%%%%%%%%%%%%%%%%%%%%%%%%%%%%%%%%%
%%                               Variables                               %%
%%%%%%%%%%%%%%%%%%%%%%%%%%%%%%%%%%%%%%%%%%%%%%%%%%%%%%%%%%%%%%%%%%%%%%%%%%%

  \subsection{Variables}

    \vspace{-1cm}
\hspace{\varindent}\begin{longtable}{|p{\varnamewidth}|p{\vardescrwidth}|l}
\cline{1-2}
\cline{1-2} \centering \textbf{Nombre} & \centering \textbf{Description}& \\
\cline{1-2}
\endhead\cline{1-2}\multicolumn{3}{r}{\small\textit{continua en la página siguiente}}\\\endfoot\cline{1-2}
\endlastfoot\raggedright \_\-\_\-p\-a\-c\-k\-a\-g\-e\-\_\-\_\- & \raggedright \textbf{Valor:} 
{\tt \texttt{'}\texttt{src}\texttt{'}}&\\
\cline{1-2}
\end{longtable}

    \index{src \textit{(package)}!src.search\_stones \textit{(module)}|)}
