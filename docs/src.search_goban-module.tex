%
% API Documentation for API Documentation
% Module src.search_goban
%
% Generated by epydoc 3.0.1
% [Wed Sep 12 04:59:27 2012]
%

%%%%%%%%%%%%%%%%%%%%%%%%%%%%%%%%%%%%%%%%%%%%%%%%%%%%%%%%%%%%%%%%%%%%%%%%%%%
%%                          Module Description                           %%
%%%%%%%%%%%%%%%%%%%%%%%%%%%%%%%%%%%%%%%%%%%%%%%%%%%%%%%%%%%%%%%%%%%%%%%%%%%

    \index{src \textit{(package)}!src.search\_goban \textit{(module)}|(}
\section{Module src.search\_goban}

    \label{src:search_goban}

%%%%%%%%%%%%%%%%%%%%%%%%%%%%%%%%%%%%%%%%%%%%%%%%%%%%%%%%%%%%%%%%%%%%%%%%%%%
%%                               Functions                               %%
%%%%%%%%%%%%%%%%%%%%%%%%%%%%%%%%%%%%%%%%%%%%%%%%%%%%%%%%%%%%%%%%%%%%%%%%%%%

  \subsection{Functions}

    \label{src:search_goban:count_perimeter}
    \index{src \textit{(package)}!src.search\_goban \textit{(module)}!src.search\_goban.count\_perimeter \textit{(function)}}

    \vspace{0.5ex}

\hspace{.8\funcindent}\begin{boxedminipage}{\funcwidth}

    \raggedright \textbf{count\_perimeter}(\textit{seq})

    \vspace{-1.5ex}

    \rule{\textwidth}{0.5\fboxrule}
\setlength{\parskip}{2ex}
Contamos el perímetro de una secuencia dada.

\setlength{\parskip}{1ex}
      \textbf{Parameters}
      \vspace{-1ex}

      \begin{quote}
        \begin{Ventry}{xxx}

          \item[seq]


secuencia de puntos
            {\it (type=CvSeq)}

        \end{Ventry}

      \end{quote}

      \textbf{Return Value}
    \vspace{-1ex}

      \begin{quote}

distancia del perímetro
      {\it (type=float)}

      \end{quote}

    \end{boxedminipage}

    \label{src:search_goban:get_corners}
    \index{src \textit{(package)}!src.search\_goban \textit{(module)}!src.search\_goban.get\_corners \textit{(function)}}

    \vspace{0.5ex}

\hspace{.8\funcindent}\begin{boxedminipage}{\funcwidth}

    \raggedright \textbf{get\_corners}(\textit{contour})

    \vspace{-1.5ex}

    \rule{\textwidth}{0.5\fboxrule}
\setlength{\parskip}{2ex}
Hallamos las esquinas a partir de un contorno y las ordenamos de la siguiente manera: ul, dl, ur, dr.  u = up, l = left, d = down, r = right.

\setlength{\parskip}{1ex}
      \textbf{Parameters}
      \vspace{-1ex}

      \begin{quote}
        \begin{Ventry}{xxxxxxx}

          \item[contour]


contorno del tablero obtenido
            {\it (type=CvSeq)}

        \end{Ventry}

      \end{quote}

      \textbf{Return Value}
    \vspace{-1ex}

      \begin{quote}

lista de esquinas
      {\it (type=list)}

      \end{quote}

    \end{boxedminipage}

    \label{src:search_goban:filter_image}
    \index{src \textit{(package)}!src.search\_goban \textit{(module)}!src.search\_goban.filter\_image \textit{(function)}}

    \vspace{0.5ex}

\hspace{.8\funcindent}\begin{boxedminipage}{\funcwidth}

    \raggedright \textbf{filter\_image}(\textit{img})

    \vspace{-1.5ex}

    \rule{\textwidth}{0.5\fboxrule}
\setlength{\parskip}{2ex}
Aplicamos unos filtros a las imágenes para facilitar su tratamiento. Buscamos contornos y suavizamos.

\setlength{\parskip}{1ex}
      \textbf{Parameters}
      \vspace{-1ex}

      \begin{quote}
        \begin{Ventry}{xxx}

          \item[img]


imagen sin filtrar
            {\it (type=CvMat)}

        \end{Ventry}

      \end{quote}

      \textbf{Return Value}
    \vspace{-1ex}

      \begin{quote}

imagen filtrada
      {\it (type=CvMat)}

      \end{quote}

    \end{boxedminipage}

    \label{src:search_goban:detect_contour}
    \index{src \textit{(package)}!src.search\_goban \textit{(module)}!src.search\_goban.detect\_contour \textit{(function)}}

    \vspace{0.5ex}

\hspace{.8\funcindent}\begin{boxedminipage}{\funcwidth}

    \raggedright \textbf{detect\_contour}(\textit{img})

    \vspace{-1.5ex}

    \rule{\textwidth}{0.5\fboxrule}
\setlength{\parskip}{2ex}
Buscamos contornos con unas características determinadas para encontrar un tablero de go en una imagen.

\setlength{\parskip}{1ex}
      \textbf{Parameters}
      \vspace{-1ex}

      \begin{quote}
        \begin{Ventry}{xxx}

          \item[img]


imagen filtrada para buscar contornos en ella
            {\it (type=CvMat)}

        \end{Ventry}

      \end{quote}

      \textbf{Return Value}
    \vspace{-1ex}

      \begin{quote}

Contorno si no lo encuentra, sino None
      {\it (type=CvSeq)}

      \end{quote}

    \end{boxedminipage}

    \label{src:search_goban:search_goban}
    \index{src \textit{(package)}!src.search\_goban \textit{(module)}!src.search\_goban.search\_goban \textit{(function)}}

    \vspace{0.5ex}

\hspace{.8\funcindent}\begin{boxedminipage}{\funcwidth}

    \raggedright \textbf{search\_goban}(\textit{img})

    \vspace{-1.5ex}

    \rule{\textwidth}{0.5\fboxrule}
\setlength{\parskip}{2ex}
Busca el tablero en una imagen.

\setlength{\parskip}{1ex}
      \textbf{Parameters}
      \vspace{-1ex}

      \begin{quote}
        \begin{Ventry}{xxx}

          \item[img]


imagen del tablero
            {\it (type=IplImage \# TODO comprobar tipo imagen)}

        \end{Ventry}

      \end{quote}

      \textbf{Return Value}
    \vspace{-1ex}

      \begin{quote}

lista de esquinas si las encuentra, sino None
      {\it (type=list or None)}

      \end{quote}

    \end{boxedminipage}


%%%%%%%%%%%%%%%%%%%%%%%%%%%%%%%%%%%%%%%%%%%%%%%%%%%%%%%%%%%%%%%%%%%%%%%%%%%
%%                               Variables                               %%
%%%%%%%%%%%%%%%%%%%%%%%%%%%%%%%%%%%%%%%%%%%%%%%%%%%%%%%%%%%%%%%%%%%%%%%%%%%

  \subsection{Variables}

    \vspace{-1cm}
\hspace{\varindent}\begin{longtable}{|p{\varnamewidth}|p{\vardescrwidth}|l}
\cline{1-2}
\cline{1-2} \centering \textbf{Name} & \centering \textbf{Description}& \\
\cline{1-2}
\endhead\cline{1-2}\multicolumn{3}{r}{\small\textit{continued on next page}}\\\endfoot\cline{1-2}
\endlastfoot\raggedright \_\-\_\-p\-a\-c\-k\-a\-g\-e\-\_\-\_\- & \raggedright \textbf{Value:} 
{\tt \texttt{'}\texttt{src}\texttt{'}}&\\
\cline{1-2}
\end{longtable}

    \index{src \textit{(package)}!src.search\_goban \textit{(module)}|)}
