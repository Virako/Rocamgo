%
% API Documentation for API Documentation
% Módulo src.cameras
%
% Generated by epydoc 3.0.1
% [Wed Sep 12 04:59:27 2012]
%

%%%%%%%%%%%%%%%%%%%%%%%%%%%%%%%%%%%%%%%%%%%%%%%%%%%%%%%%%%%%%%%%%%%%%%%%%%%
%%                          Módulo Description                           %%
%%%%%%%%%%%%%%%%%%%%%%%%%%%%%%%%%%%%%%%%%%%%%%%%%%%%%%%%%%%%%%%%%%%%%%%%%%%

    \index{src \textit{(package)}!src.cameras \textit{(module)}|(}
\section{Módulo src.cameras}

    \label{src:cameras}

%%%%%%%%%%%%%%%%%%%%%%%%%%%%%%%%%%%%%%%%%%%%%%%%%%%%%%%%%%%%%%%%%%%%%%%%%%%
%%                               Variables                               %%
%%%%%%%%%%%%%%%%%%%%%%%%%%%%%%%%%%%%%%%%%%%%%%%%%%%%%%%%%%%%%%%%%%%%%%%%%%%

  \subsection{Variables}

    \vspace{-1cm}
\hspace{\varindent}\begin{longtable}{|p{\varnamewidth}|p{\vardescrwidth}|l}
\cline{1-2}
\cline{1-2} \centering \textbf{Nombre} & \centering \textbf{Description}& \\
\cline{1-2}
\endhead\cline{1-2}\multicolumn{3}{r}{\small\textit{continua en la página siguiente}}\\\endfoot\cline{1-2}
\endlastfoot\raggedright \_\-\_\-p\-a\-c\-k\-a\-g\-e\-\_\-\_\- & \raggedright \textbf{Valor:} 
{\tt \texttt{'}\texttt{src}\texttt{'}}&\\
\cline{1-2}
\raggedright c\-a\-m\-e\-r\-a\- & \raggedright cámara seleccionada

            {\it (tipo=Capture)}&\\
\cline{1-2}
\raggedright c\-a\-m\-e\-r\-a\-s\- & \raggedright lista de cámaras

            {\it (tipo=list)}&\\
\cline{1-2}
\end{longtable}


%%%%%%%%%%%%%%%%%%%%%%%%%%%%%%%%%%%%%%%%%%%%%%%%%%%%%%%%%%%%%%%%%%%%%%%%%%%
%%                           Clase Description                           %%
%%%%%%%%%%%%%%%%%%%%%%%%%%%%%%%%%%%%%%%%%%%%%%%%%%%%%%%%%%%%%%%%%%%%%%%%%%%

    \index{src \textit{(package)}!src.cameras \textit{(module)}!src.cameras.Cameras \textit{(class)}|(}
\subsection{Clase Cameras}

    \label{src:cameras:Cameras}

Clase para abrir las cámaras disponibles en el ordenador.

%%%%%%%%%%%%%%%%%%%%%%%%%%%%%%%%%%%%%%%%%%%%%%%%%%%%%%%%%%%%%%%%%%%%%%%%%%%
%%                                Métodos                                %%
%%%%%%%%%%%%%%%%%%%%%%%%%%%%%%%%%%%%%%%%%%%%%%%%%%%%%%%%%%%%%%%%%%%%%%%%%%%

  \subsubsection{Métodos}

    \label{src:cameras:Cameras:__init__}
    \index{src \textit{(package)}!src.cameras \textit{(module)}!src.cameras.Cameras \textit{(class)}!src.cameras.Cameras.\_\_init\_\_ \textit{(method)}}

    \vspace{0.5ex}

\hspace{.8\funcindent}\begin{boxedminipage}{\funcwidth}

    \raggedright \textbf{\_\_init\_\_}(\textit{self})

\setlength{\parskip}{2ex}
\setlength{\parskip}{1ex}
    \end{boxedminipage}

    \label{src:cameras:Cameras:on_mouse}
    \index{src \textit{(package)}!src.cameras \textit{(module)}!src.cameras.Cameras \textit{(class)}!src.cameras.Cameras.on\_mouse \textit{(method)}}

    \vspace{0.5ex}

\hspace{.8\funcindent}\begin{boxedminipage}{\funcwidth}

    \raggedright \textbf{on\_mouse}(\textit{self}, \textit{event}, \textit{x}, \textit{y}, \textit{flags}, \textit{camera})

    \vspace{-1.5ex}

    \rule{\textwidth}{0.5\fboxrule}
\setlength{\parskip}{2ex}
Capturador de eventos de click de ratón.

\setlength{\parskip}{1ex}
      \textbf{Parametros}
      \vspace{-1ex}

      \begin{quote}
        \begin{Ventry}{xxxxxx}

          \item[event]


Evento capturado.
            {\it (tipo=int)}

          \item[x]


posición x del ratón.
            {\it (tipo=int)}

          \item[y]


posición y del ratón.
            {\it (tipo=int)}

          \item[camera]


objeto Camera.
            {\it (tipo=Camera)}

        \end{Ventry}

      \end{quote}

    \end{boxedminipage}

    \label{src:cameras:Cameras:check_cameras}
    \index{src \textit{(package)}!src.cameras \textit{(module)}!src.cameras.Cameras \textit{(class)}!src.cameras.Cameras.check\_cameras \textit{(method)}}

    \vspace{0.5ex}

\hspace{.8\funcindent}\begin{boxedminipage}{\funcwidth}

    \raggedright \textbf{check\_cameras}(\textit{self}, \textit{num}={\tt 99})

    \vspace{-1.5ex}

    \rule{\textwidth}{0.5\fboxrule}
\setlength{\parskip}{2ex}
Comprueba las cámaras disponibles.

\setlength{\parskip}{1ex}
      \textbf{Parametros}
      \vspace{-1ex}

      \begin{quote}
        \begin{Ventry}{xxx}

          \item[num]


máximo número de cámaras a comprobar
          \item[num]


99 por defecto, ya que en Linux es lo permitido
          \item[num]


int
        \end{Ventry}

      \end{quote}

      \textbf{Return Valor}
    \vspace{-1ex}

      \begin{quote}

lista de cámaras disponibles
      {\it (tipo=list of Camera)}

      \end{quote}

    \end{boxedminipage}

    \label{src:cameras:Cameras:show_and_select_camera}
    \index{src \textit{(package)}!src.cameras \textit{(module)}!src.cameras.Cameras \textit{(class)}!src.cameras.Cameras.show\_and\_select\_camera \textit{(method)}}

    \vspace{0.5ex}

\hspace{.8\funcindent}\begin{boxedminipage}{\funcwidth}

    \raggedright \textbf{show\_and\_select\_camera}(\textit{self})

    \vspace{-1.5ex}

    \rule{\textwidth}{0.5\fboxrule}
\setlength{\parskip}{2ex}
Muestra las cámaras disponibles en ventanas y da la opción de seleccionar una de ellas pulsando doble click.

\setlength{\parskip}{1ex}
      \textbf{Return Valor}
    \vspace{-1ex}

      \begin{quote}

cámara seleccionada
      {\it (tipo=Camera)}

      \end{quote}

    \end{boxedminipage}

    \index{src \textit{(package)}!src.cameras \textit{(module)}!src.cameras.Cameras \textit{(class)}|)}
    \index{src \textit{(package)}!src.cameras \textit{(module)}|)}
