%
% API Documentation for API Documentation
% Module src.goban
%
% Generated by epydoc 3.0.1
% [Wed Sep 12 04:59:27 2012]
%

%%%%%%%%%%%%%%%%%%%%%%%%%%%%%%%%%%%%%%%%%%%%%%%%%%%%%%%%%%%%%%%%%%%%%%%%%%%
%%                          Module Description                           %%
%%%%%%%%%%%%%%%%%%%%%%%%%%%%%%%%%%%%%%%%%%%%%%%%%%%%%%%%%%%%%%%%%%%%%%%%%%%

    \index{src \textit{(package)}!src.goban \textit{(module)}|(}
\section{Module src.goban}

    \label{src:goban}

%%%%%%%%%%%%%%%%%%%%%%%%%%%%%%%%%%%%%%%%%%%%%%%%%%%%%%%%%%%%%%%%%%%%%%%%%%%
%%                               Variables                               %%
%%%%%%%%%%%%%%%%%%%%%%%%%%%%%%%%%%%%%%%%%%%%%%%%%%%%%%%%%%%%%%%%%%%%%%%%%%%

  \subsection{Variables}

    \vspace{-1cm}
\hspace{\varindent}\begin{longtable}{|p{\varnamewidth}|p{\vardescrwidth}|l}
\cline{1-2}
\cline{1-2} \centering \textbf{Name} & \centering \textbf{Description}& \\
\cline{1-2}
\endhead\cline{1-2}\multicolumn{3}{r}{\small\textit{continued on next page}}\\\endfoot\cline{1-2}
\endlastfoot\raggedright \_\-\_\-p\-a\-c\-k\-a\-g\-e\-\_\-\_\- & \raggedright \textbf{Value:} 
{\tt \texttt{'}\texttt{src}\texttt{'}}&\\
\cline{1-2}
\raggedright g\-o\-b\-a\-n\- & \raggedright matriz de piedras puestas

            {\it (type=list)}&\\
\cline{1-2}
\raggedright i\-g\-s\- & \raggedright Objeto Igs

            {\it (type=Igs)}&\\
\cline{1-2}
\raggedright k\-i\-f\-u\- & \raggedright Objeto Kifu

            {\it (type=Kifu)}&\\
\cline{1-2}
\raggedright s\-t\-a\-t\-i\-s\-t\-i\-c\-a\-l\- & \raggedright matriz de estadísticas para comprobar piedras buenas o malas

            {\it (type=list)}&\\
\cline{1-2}
\raggedright s\-t\-o\-n\-e\-s\- & \raggedright piedras a comprobar para añadir a estadísticas

            {\it (type=list)}&\\
\cline{1-2}
\end{longtable}


%%%%%%%%%%%%%%%%%%%%%%%%%%%%%%%%%%%%%%%%%%%%%%%%%%%%%%%%%%%%%%%%%%%%%%%%%%%
%%                           Class Description                           %%
%%%%%%%%%%%%%%%%%%%%%%%%%%%%%%%%%%%%%%%%%%%%%%%%%%%%%%%%%%%%%%%%%%%%%%%%%%%

    \index{src \textit{(package)}!src.goban \textit{(module)}!src.goban.Goban \textit{(class)}|(}
\subsection{Class Goban}

    \label{src:goban:Goban}

Clase tablero, contiene la matriz de estadíticas y funciones para rellenar el tablero.

%%%%%%%%%%%%%%%%%%%%%%%%%%%%%%%%%%%%%%%%%%%%%%%%%%%%%%%%%%%%%%%%%%%%%%%%%%%
%%                                Methods                                %%
%%%%%%%%%%%%%%%%%%%%%%%%%%%%%%%%%%%%%%%%%%%%%%%%%%%%%%%%%%%%%%%%%%%%%%%%%%%

  \subsubsection{Methods}

    \label{src:goban:Goban:__init__}
    \index{src \textit{(package)}!src.goban \textit{(module)}!src.goban.Goban \textit{(class)}!src.goban.Goban.\_\_init\_\_ \textit{(method)}}

    \vspace{0.5ex}

\hspace{.8\funcindent}\begin{boxedminipage}{\funcwidth}

    \raggedright \textbf{\_\_init\_\_}(\textit{self}, \textit{size})

    \vspace{-1.5ex}

    \rule{\textwidth}{0.5\fboxrule}
\setlength{\parskip}{2ex}
Crea dos matrices de tamaño pasado por parámetro, una para estadísticas y otra para guardar el estado de las piedras. Creamos un set de piedras para ir guardando las piedras que estemos comprobando.  También inicializa un kifu para guardar la partida y un el objetos igs que se encargará de conectarse con el servidor que subirá la partida.

\setlength{\parskip}{1ex}
      \textbf{Parameters}
      \vspace{-1ex}

      \begin{quote}
        \begin{Ventry}{xxxx}

          \item[size]


tamaño del tablero
            {\it (type=int)}

        \end{Ventry}

      \end{quote}

    \end{boxedminipage}

    \label{src:goban:Goban:add_stones_to_statistical}
    \index{src \textit{(package)}!src.goban \textit{(module)}!src.goban.Goban \textit{(class)}!src.goban.Goban.add\_stones\_to\_statistical \textit{(method)}}

    \vspace{0.5ex}

\hspace{.8\funcindent}\begin{boxedminipage}{\funcwidth}

    \raggedright \textbf{add\_stones\_to\_statistical}(\textit{self}, \textit{stones})

    \vspace{-1.5ex}

    \rule{\textwidth}{0.5\fboxrule}
\setlength{\parskip}{2ex}
Recorremos la lista de piedras pasadas por parámetros para buscar hacer comprobaciones estadísticas en esas piedras, luego recorremos la lista de piedras guardada y la actualizamos. Actualiza kifu, igs y el tablero donde guardamos el estado de las piedras cuando detecta estadísticamente que una piedra se ha puesto.

\setlength{\parskip}{1ex}
      \textbf{Parameters}
      \vspace{-1ex}

      \begin{quote}
        \begin{Ventry}{xxxxxx}

          \item[stones]


lista de piedras
            {\it (type=list)}

        \end{Ventry}

      \end{quote}

    \end{boxedminipage}

    \label{src:goban:Goban:print_st}
    \index{src \textit{(package)}!src.goban \textit{(module)}!src.goban.Goban \textit{(class)}!src.goban.Goban.print\_st \textit{(method)}}

    \vspace{0.5ex}

\hspace{.8\funcindent}\begin{boxedminipage}{\funcwidth}

    \raggedright \textbf{print\_st}(\textit{self})

\setlength{\parskip}{2ex}
\setlength{\parskip}{1ex}
    \end{boxedminipage}

    \label{src:goban:Goban:__str__}
    \index{src \textit{(package)}!src.goban \textit{(module)}!src.goban.Goban \textit{(class)}!src.goban.Goban.\_\_str\_\_ \textit{(method)}}

    \vspace{0.5ex}

\hspace{.8\funcindent}\begin{boxedminipage}{\funcwidth}

    \raggedright \textbf{\_\_str\_\_}(\textit{self})

\setlength{\parskip}{2ex}
\setlength{\parskip}{1ex}
    \end{boxedminipage}

    \index{src \textit{(package)}!src.goban \textit{(module)}!src.goban.Goban \textit{(class)}|)}
    \index{src \textit{(package)}!src.goban \textit{(module)}|)}
