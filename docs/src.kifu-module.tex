%
% API Documentation for API Documentation
% Módulo src.kifu
%
% Generated by epydoc 3.0.1
% [Wed Sep 12 04:59:27 2012]
%

%%%%%%%%%%%%%%%%%%%%%%%%%%%%%%%%%%%%%%%%%%%%%%%%%%%%%%%%%%%%%%%%%%%%%%%%%%%
%%                          Módulo Descripción                           %%
%%%%%%%%%%%%%%%%%%%%%%%%%%%%%%%%%%%%%%%%%%%%%%%%%%%%%%%%%%%%%%%%%%%%%%%%%%%

    \index{src \textit{(package)}!src.kifu \textit{(module)}|(}
\section{Módulo src.kifu}

    \label{src:kifu}

%%%%%%%%%%%%%%%%%%%%%%%%%%%%%%%%%%%%%%%%%%%%%%%%%%%%%%%%%%%%%%%%%%%%%%%%%%%
%%                               Variables                               %%
%%%%%%%%%%%%%%%%%%%%%%%%%%%%%%%%%%%%%%%%%%%%%%%%%%%%%%%%%%%%%%%%%%%%%%%%%%%

  \subsection{Variables}

    \vspace{-1cm}
\hspace{\varindent}\begin{longtable}{|p{\varnamewidth}|p{\vardescrwidth}|l}
\cline{1-2}
\cline{1-2} \centering \textbf{Nombre} & \centering \textbf{Descripción}& \\
\cline{1-2}
\endhead\cline{1-2}\multicolumn{3}{r}{\small\textit{continúa en la página siguiente}}\\\endfoot\cline{1-2}
\endlastfoot\raggedright \_\-\_\-p\-a\-c\-k\-a\-g\-e\-\_\-\_\- & \raggedright \textbf{Valor:} 
{\tt \texttt{'}\texttt{src}\texttt{'}}&\\
\cline{1-2}
\raggedright d\-i\-r\- & \raggedright dirección del directorio donde guardaremos la partida

            {\it (tipo=str)}&\\
\cline{1-2}
\raggedright h\-a\-n\-d\-i\-c\-a\-p\- & \raggedright numero de piedras de ventaja

            {\it (tipo=int)}&\\
\cline{1-2}
\raggedright n\-u\-m\-\_\-j\-u\-g\- & \raggedright número de jugada actual

            {\it (tipo=int)}&\\
\cline{1-2}
\raggedright p\-a\-t\-h\- & \raggedright directorio donde guardaremos las partidas

            {\it (tipo=str)}&\\
\cline{1-2}
\raggedright p\-l\-a\-y\-e\-r\-1\- & \raggedright nombre del jugador 1

            {\it (tipo=str)}&\\
\cline{1-2}
\raggedright p\-l\-a\-y\-e\-r\-2\- & \raggedright nombre del jugador 2

            {\it (tipo=str)}&\\
\cline{1-2}
\raggedright r\-a\-n\-k\-\_\-p\-l\-a\-y\-e\-r\-1\- & \raggedright nivel del jugador 1

            {\it (tipo=str)}&\\
\cline{1-2}
\raggedright r\-a\-n\-k\-\_\-p\-l\-a\-y\-e\-r\-2\- & \raggedright nivel del jugador 2

            {\it (tipo=str)}&\\
\cline{1-2}
\end{longtable}


\newpage
%%%%%%%%%%%%%%%%%%%%%%%%%%%%%%%%%%%%%%%%%%%%%%%%%%%%%%%%%%%%%%%%%%%%%%%%%%%
%%                           Clase Descripción                           %%
%%%%%%%%%%%%%%%%%%%%%%%%%%%%%%%%%%%%%%%%%%%%%%%%%%%%%%%%%%%%%%%%%%%%%%%%%%%

    \index{src \textit{(package)}!src.kifu \textit{(module)}!src.kifu.Kifu \textit{(class)}|(}
\subsection{Clase Kifu}

    \label{src:kifu:Kifu}

Clase para crear un fichero .sgf y guardar la partida.

%%%%%%%%%%%%%%%%%%%%%%%%%%%%%%%%%%%%%%%%%%%%%%%%%%%%%%%%%%%%%%%%%%%%%%%%%%%
%%                                Métodos                                %%
%%%%%%%%%%%%%%%%%%%%%%%%%%%%%%%%%%%%%%%%%%%%%%%%%%%%%%%%%%%%%%%%%%%%%%%%%%%

  \subsubsection{Métodos}

    \label{src:kifu:Kifu:__init__}
    \index{src \textit{(package)}!src.kifu \textit{(module)}!src.kifu.Kifu \textit{(class)}!src.kifu.Kifu.\_\_init\_\_ \textit{(method)}}

    \vspace{0.5ex}

\hspace{.8\funcindent}\begin{boxedminipage}{\funcwidth}

    \raggedright \textbf{\_\_init\_\_}(\textit{self}, \textit{player1}={\tt \texttt{'}\texttt{j1}\texttt{'}}, \textit{player2}={\tt \texttt{'}\texttt{j2}\texttt{'}}, \textit{handicap}={\tt 0}, \textit{path}={\tt \texttt{'}\texttt{sgf}\texttt{'}}, \textit{rank\_player1}={\tt \texttt{'}\texttt{20k}\texttt{'}}, \textit{rank\_player2}={\tt \texttt{'}\texttt{20k}\texttt{'}})

    \vspace{-1.5ex}

    \rule{\textwidth}{0.5\fboxrule}
\setlength{\parskip}{2ex}
Inicializamos configuración del archivo sgf.

\setlength{\parskip}{1ex}
      \textbf{Parámetros}
      \vspace{-1ex}

      \begin{quote}
        \begin{Ventry}{xxxxxxxxxxxx}

          \item[player1]


nombre del jugador 1
            {\it (tipo=str)}

          \item[player1]


j1 por defecto
            {\it (tipo=str)}

          \item[player2]


nombre del jugador 2
            {\it (tipo=str)}

          \item[player2]


j2 por defecto
            {\it (tipo=str)}

          \item[handicap]


handicap dado en la partida
            {\it (tipo=int)}

          \item[handicap]


ninguno por defecto (0)
            {\it (tipo=int)}

          \item[path]


ruta relativa donde guardamos el fichero
            {\it (tipo=str)}

          \item[path]


carpeta sgf por defecto
            {\it (tipo=str)}

          \item[rank\_player1]


rango del jugador 1
            {\it (tipo=str)}

          \item[rank\_player1]


20k por defecto, nivel de inicio en el go
            {\it (tipo=str)}

          \item[rank\_player2]


rango del jugador 2
            {\it (tipo=str)}

          \item[rank\_player2]


20k por defecto, nivel de inicio en el go
            {\it (tipo=str)}

        \end{Ventry}

      \end{quote}

    \end{boxedminipage}

    \label{src:kifu:Kifu:add_stone}
    \index{src \textit{(package)}!src.kifu \textit{(module)}!src.kifu.Kifu \textit{(class)}!src.kifu.Kifu.add\_stone \textit{(method)}}

    \vspace{0.5ex}

\hspace{.8\funcindent}\begin{boxedminipage}{\funcwidth}

    \raggedright \textbf{add\_stone}(\textit{self}, \textit{pos}, \textit{color})

    \vspace{-1.5ex}

    \rule{\textwidth}{0.5\fboxrule}
\setlength{\parskip}{2ex}
Añadir piedra al sgf.

\setlength{\parskip}{1ex}
      \textbf{Parámetros}
      \vspace{-1ex}

      \begin{quote}
        \begin{Ventry}{xxxxx}

          \item[pos]


posición de la piedra
            {\it (tipo=tuple)}

          \item[color]


color de la piedra
            {\it (tipo=int)}

        \end{Ventry}

      \end{quote}

    \end{boxedminipage}

    \label{src:kifu:Kifu:end_file}
    \index{src \textit{(package)}!src.kifu \textit{(module)}!src.kifu.Kifu \textit{(class)}!src.kifu.Kifu.end\_file \textit{(method)}}

    \vspace{0.5ex}

\hspace{.8\funcindent}\begin{boxedminipage}{\funcwidth}

    \raggedright \textbf{end\_file}(\textit{self})

    \vspace{-1.5ex}

    \rule{\textwidth}{0.5\fboxrule}
\setlength{\parskip}{2ex}
Cerrar el fichero y dejarlo listo para poder abrirlo.

\setlength{\parskip}{1ex}
    \end{boxedminipage}

    \index{src \textit{(package)}!src.kifu \textit{(module)}!src.kifu.Kifu \textit{(class)}|)}
    \index{src \textit{(package)}!src.kifu \textit{(module)}|)}
