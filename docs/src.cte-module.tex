%
% API Documentation for API Documentation
% Módulo src.cte
%
% Generated by epydoc 3.0.1
% [Wed Sep 12 04:59:27 2012]
%

%%%%%%%%%%%%%%%%%%%%%%%%%%%%%%%%%%%%%%%%%%%%%%%%%%%%%%%%%%%%%%%%%%%%%%%%%%%
%%                          Módulo Description                           %%
%%%%%%%%%%%%%%%%%%%%%%%%%%%%%%%%%%%%%%%%%%%%%%%%%%%%%%%%%%%%%%%%%%%%%%%%%%%

    \index{src \textit{(package)}!src.cte \textit{(module)}|(}
\section{Módulo src.cte}

    \label{src:cte}

%%%%%%%%%%%%%%%%%%%%%%%%%%%%%%%%%%%%%%%%%%%%%%%%%%%%%%%%%%%%%%%%%%%%%%%%%%%
%%                               Variables                               %%
%%%%%%%%%%%%%%%%%%%%%%%%%%%%%%%%%%%%%%%%%%%%%%%%%%%%%%%%%%%%%%%%%%%%%%%%%%%

  \subsection{Variables}

    \vspace{-1cm}
\hspace{\varindent}\begin{longtable}{|p{\varnamewidth}|p{\vardescrwidth}|l}
\cline{1-2}
\cline{1-2} \centering \textbf{Nombre} & \centering \textbf{Description}& \\
\cline{1-2}
\endhead\cline{1-2}\multicolumn{3}{r}{\small\textit{continua en la página siguiente}}\\\endfoot\cline{1-2}
\endlastfoot\raggedright N\-U\-M\-\_\-E\-D\-G\-E\-S\- & \raggedright número de esquinas que existen en un tablero

\textbf{Valor:} 
{\tt 4}            {\it (tipo=int)}&\\
\cline{1-2}
\raggedright R\-E\-L\-A\-T\-I\-O\-N\-\_\-W\-E\-I\-G\-H\-T\-\_\-H\-E\-I\-G\-H\-T\- & \raggedright relación anchura/altura que existe en el tablero

\textbf{Valor:} 
{\tt 0.933333333333}            {\it (tipo=float)}&\\
\cline{1-2}
\raggedright M\-A\-X\-\_\-C\-A\-M\-E\-R\-A\-S\- & \raggedright número máximo de cámaras a buscar

\textbf{Valor:} 
{\tt 99}            {\it (tipo=int)}&\\
\cline{1-2}
\raggedright G\-O\-B\-A\-N\-\_\-S\-I\-Z\-E\- & \raggedright tamaño del tablero

\textbf{Valor:} 
{\tt 19}            {\it (tipo=int)}&\\
\cline{1-2}
\raggedright B\-L\-A\-C\-K\- & \raggedright constante para decir que una piedra es negra

\textbf{Valor:} 
{\tt 1}            {\it (tipo=int)}&\\
\cline{1-2}
\raggedright W\-H\-I\-T\-E\- & \raggedright constante para decir que una piedra es blanca

\textbf{Valor:} 
{\tt 2}            {\it (tipo=int)}&\\
\cline{1-2}
\raggedright c\-u\-r\-r\-e\-n\-t\-\_\-d\-a\-t\-e\- & \raggedright \textbf{Valor:} 
{\tt \texttt{'}\texttt{12 Sep 2012}\texttt{'}}&\\
\cline{1-2}
\raggedright H\-E\-A\-D\-E\-R\-\_\-S\-G\-F\- & \raggedright %
\begin{itemize}

\item AB: Add Black.

\item AW: Add White.

\item AN: Annotations.

\item AP: Application.

\item B: a move by Black at the location specified by the property value.

\item BR: Black Rank.

\item BT: Black Team.

\item C: Comment.

\item CP: Copyright. See Kifu Copyright Discussion.

\item DT: Date.

\item EV: Event.

\item FF: File format.

\item GM: Game.

\item GN: Game Nombre.

\item HA: Handicap.

\item KM: Komi.

\item ON: Opening.

\item OT: Overtime.

\item PB: Black Nombre.

\item PC: Place.

\item PL: Player.

\item PW: White Nombre.

\item RE: Result.

\item RO: Round.

\item RU: Rules.

\item SO: Source.

\item SZ: Size.

\item TM: Time limit.

\item US: User.

\item W: a move by White at the location specified by the property value.

\item WR: White Rank.

\item WT: White Team.

\end{itemize}

\textbf{Valor:} 
{\tt \texttt{[}\texttt{'}\texttt{(;FF[4]GM[1]SZ[19]}\texttt{'}\texttt{, }\texttt{'}\texttt{{\textbackslash}nAP[Rocamgo]}\texttt{'}\texttt{, }\texttt{'}\texttt{{\textbackslash}nHA[0]}\texttt{'}\texttt{, }\texttt{'}\texttt{{\textbackslash}nKM[}\texttt{...}}            {\it (tipo=str)}&\\
\cline{1-2}
\raggedright \_\-\_\-p\-a\-c\-k\-a\-g\-e\-\_\-\_\- & \raggedright \textbf{Valor:} 
{\tt \texttt{'}\texttt{src}\texttt{'}}&\\
\cline{1-2}
\end{longtable}

    \index{src \textit{(package)}!src.cte \textit{(module)}|)}
